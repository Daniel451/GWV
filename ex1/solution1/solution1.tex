\documentclass[10pt,a4paper]{article}
\usepackage[utf8]{inputenc}
\usepackage{amsmath}
\usepackage{amsfonts}
\usepackage{amssymb}
\usepackage{amsthm}
\usepackage{polynom} % Polynomdivision
\usepackage{url,xspace,boxedminipage}   % Accurate display of URLs
\usepackage{csquotes} % Quotes



\title{\textbf{\huge Grundlagen der Wissensverarbeitung
\\\Large Blatt 1}}
\author{Daniel Speck, Lena Niermeyer, Florian Klückman}
\date{15.10.2015}

\setlength{\topmargin}{-1.0cm}
\setlength{\textheight}{650pt}


\begin{document}


	% Titel, Autor & Abgabedatum
	\maketitle




	\section*{Aufgabe 1.1 (Application Scenarios for Artificial Intelligence)}
	
		\subsection*{1. Navigating through a labyrinth}
		
			\textbf{Task \& result:} An agent has to find a way through a labyrinth. The found solution should be as efficient as possible.
		
		\subsection*{2. Calibration of sensor data}
		
			\textbf{Task \& result:} Real world data recorded from sensors often contains noise and other distortions of input data. A supervised agent could be trained to find patterns in those distortions and deliver more distortion-free data.
		
		\subsection*{3. Image classification}
		
			\textbf{Task \& result:} Much information is meaningless if it is not accessible easily for humans or computers. Large databases of images (e.g. Google images) have to be categorized correctly to build up a useful database. An agent could be used to classify this data automatically.
	
	\section*{Aufgabe 1.2 (AI Terminology)}

		\textit{Information} is a subset of \textit{knowledge} created as well as spread via sharing. Machines and computer programs may be senders or receivers of information. The world wide web contains information in the form of messages, texts, encyclopedias, blogs, libraries and so forth. Not all information is \textit{explicit knowledge}, but \textit{explicit knowledge} consists of \enquote{saved}, well formulated information. 
		\\

		%Information ist ein Teilmenge vom Wissen, die durch Teilen vervielfältigt wird. Maschinen und Computerprogramme können sowohl Sender als auch Empfänger von Informationen sein. Das Internet enthält Informationen in Form von Nachrichten und Bibliotheken, ein Smartphone enthält technische Details zu sich und zum Datenverbrauch. Explizites Wissen wird als Information gespeichert.
		
		\noindent \textit{Implicit knowledge} is unconsciously, it can not be or has not been articulated. An individual \enquote{carries} \textit{implicit knowledge} with its skills and abilities. It is (logically) deducible from observable information or behavior and it implies itself with the existence of such information, behavior, skills, abilities, \dots
		\\
		A popular German saying is \enquote{zwischen den Zeilen lesen} which means \enquote{read between the lines}, this saying indicates the existence of hidden knowledge, \textit{implicit knowledge}, which has to be discovered by observation and/or deductions.
		\\
		Examples for skills and abilities would be bicycling, swimming or socialization in communities. \textit{Implicit knowledge} can also be more technical, e.g. in the form of data in data warehouses or databases of companies. \enquote{Here} the \textit{implicit knowledge} would not be the data itself (which would be considered \textit{explicit knowledge}) but conclusions which could be drawn out of this information. A concrete example could be the logging of each access of a specific information with the lack of further interpretations of this logged data (so it would remain a simple counter).
		\\
		% Implizites Wissen ist unbewusstes Wissen, das man nicht artikulieren kann, weil es im Können eines Individuums steckt und somit nicht kommuniziert werden kann. Fahrradfahren oder die Sozialisierung in den Unternehmensalltag sind Beispiele.		Unternehmen liegt impl. Wissen in Form von Betriebsdatenbanken und Data Warehouses vor. Hier werden impl. Daten, wie zB Häufigkeit von Zugriffen gespeichert, aber keine weiteren Informationen aus diesen Daten gezogen. Würde man das tun und die Daten nutzbar machen, wären es explizite Daten.
		
		\noindent \textit{Explicit knowledge} is articulated information in a well formulated structure which could be texts, articles, formulas, algorithms, computer programs, rules, laws, diagrams and other forms of held knowledge. A book of mathematics for example holds formulas and therefore unambiguous information. 
		\\
		%Explizites Wissen ist artikuliertes und in Form von Texten, Diagrammen usw. Festgehaltenes Wissen. Beispiele hierfür sind Produkt Spezifikationen, wissenschaftliche Formeln oder Computer Programme oder Regeln, auf die man sich geeinigt hat (Prüfungsordnungen, Rechtschreibung).
		% Ein Mathebuch enthält Formeln, also eindeutig kommuniziertes Wissen.
		% Auf dem Laptop ist explizites Wissen in Form von Hilfeseiten (Linux Man Pages) gespeichert und über das Internet greift man auf explizites Wissen, gespeichert auf Servern, zu, zB den Texten auf Wikipedia.
		
		%Quelle: \url{http://www.nickols.us/Knowledge_in_KM.htm}
		
		
		\noindent \textit{fully observable} versus \textit{partially observable}:
		\\ 
		%In einer voll beobachtbaren Umgebung ist der Agent jederzeit ausreichend informiert, um eine optimale Entscheidung zu treffen. In anderen Umgebungen müssen Agenten auf Erfahrungen oder Erinnerungen zurückgreifen, um die beste Entscheidung zu treffen.
		%Das Damespiel ist zB voll beobachtbar, während Poker nur teilweise beobachtbar ist. Möchte man für diese Spiele einen Algorithmus schreiben, um Entscheidungen nicht intuitiv, sondern logisch zu treffen, so ist beim Damespiel für jede Stellung auf dem Brett (die ja jederzeit sichtbar ist), ein Gegenschlag durch den Algorithmus eindeutig definierbar. Beim Poker ist unklar, welche Karten andere Spieler auf der Hand haben und somit kann aufgrund dieser fehlenden Information keine optimale Lösung gefunden werden.
		In a fully observable environment the agent is always sufficiently informed to make an optimal decision. In other environments agents must rely on experience or memories in order to make the best decision.
		Checkers eg is fully observable, while poker is only partially observable. If you want to write an algorithm for these Games to make no intuitive decisions, but logical, then in checkers for each position on the board (which are visible at all times), a reaction by the algorithm is clearly defined. In poker, the hand/ cards of other players are unclear and because of this lack of information no optimal solution can be found.
Reason, why the difference might be important when designing AI applications for the given environment:
If there are unobserved states, it is much harder to learn a good policy. Unobserved states are a lot more difficult to learn a good policy.
%Quelle: http://metaoptimize.com/qa/questions/9924/rl-difference-between-partially-observable-and-a-stochastic-environment
		\noindent \textit{discrete} versus \textit{continuous}:
		\\ 
		 
		%Diskret meint, dass verschiedene Werte verschieden Bedeutungen haben. Beim Schach, wo man ein 8x8 Array hat, das verschiedene Werte represäntiert: 1=Turm, 2=Pferd, 3=Läufer usw. 2 ist dann etwas anderes als 1 und bedeutet nicht 2xTurm.
	%	Kontinuierlich meint, dass die Werte zwar die gleiche Bedeutung, aber verschiedene Höhen haben. ZB Pixelwerte: 0 bedeutet “kein Licht” (schwarz) und 255 bedeutet “maximal viel Licht” (weiss). 
	%	Problem bei diskreter Umgebung ist, dass man stark abstrahieren muss für diesen Fall, den die meisten Dinge in der Natur sind kontinuierlich (Temperatur, Gesundheit usw.) und nicht so einfach einzuteilen. Andererseits erlaubt diese Absatraktion eine Handhabbarkeit mit den Dingen (Lichtschalter: An, Aus ist einfach. Dimmbar wäre dann kontinuierlich und erfordert weitere Entscheidungen, was alles komplexer macht).
		Discrete means that different values have different meanings. In chess, where you have an 8x8 array representing different values: 1 = pawn 2 = rook , 3 = bishop.... 2 then is something else than 1 and does not mean 2xpawns.
Continuous means that the values have the same meaning, but different amounts. For example, pixel values: 0 means "no light" (black) and 255 means "maximum amount of light" (white).

Problem in discrete environment is that you have to think abstract, but most things in nature are continuous (temperature , health, etc.) and not so simple. On the other hand, abstraction allows manageability of things (light switch : On, Off is simply, Dimmable would be continuously and requires further decisions, which makes everything more complex).

Reason, why the difference might be important when designing AI applications for the given environment:
The difference is important as many statistical and data mining algorithms can handle one type but not the other. For example in regular regression, the Y must be continuous. In logistic regression the Y is discrete.
%http://stats.stackexchange.com/questions/206/what-is-the-difference-between-discrete-data-and-continuous-data


		
		\noindent \textit{deterministic} versus \textit{stochastic}:
		\\ 
		%Deterministisch bedeutet, dass beim selben Dateninput immer die gleichen Ergebnisse herauskommen, da von mathematischen Gesetzen ausgegangen wird.
		%Bei der stochastischen Umgebungen wird mit Wahrscheinlichkeitswerten gerechnet, so dass die gleiche Eingabe zu einem anderen Ergebnis führen kann.
		%Kreuzworträtsel sind zB deterministisch, während medizinische Diagnosen stochastisch sind. Problem bei determ. Umfeld: unflexibel und es kann mit Fehlern weiter gerechnet werden, weil das System so starr ist. Problem bei stoch. Umfeld: uneindeutig. Für Planung und Folgeschritte sind präzise Werte / Informationen hilfreich, doch durch die Wahrscheinlichkeitsverteilung kommen stets mehrere Daten zur Weiterverarbeitung in Frage.
		
		Deterministic means that the same input results in the same output, through mathematical laws are assumed.
In stochastic environments probability values are part of the calculation, so that the same input can result in different outputs.
Crosswords e.g. are deterministic, while medical diagnoses are stochastically. Problem with determ. Environment: inflexible and further calculation with wrong results, because the system is so rigid. Problem at Stoch. Environment: ambiguous. For planning and follow-up steps precise values/information is helpful, but through probability values much data can not be used for further processing.

Reason, why the difference might be important when designing AI applications for the given environment:
Modeling and simulating a stochastic process is more difficult and in complex systems you may have to neglect some variables to be able to use the model.
%Quelle: http://www.edaboard.com/thread26658.html



		
		
		% Quellen:
		% - https://www.udacity.com/wiki/cs271/unit1-notes
		% - http://www.aiqus.com/forum/questions/2539/discrete-vs-continuous
		

\end{document}





