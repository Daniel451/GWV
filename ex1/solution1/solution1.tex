\documentclass[10pt,a4paper]{article}
\usepackage[utf8]{inputenc}
\usepackage{amsmath}
\usepackage{amsfonts}
\usepackage{amssymb}
\usepackage{amsthm}
\usepackage{polynom} % Polynomdivision



\title{\textbf{\huge Grundlagen der Wissensverarbeitung
\\\Large Blatt 1}}
\author{Daniel Speck, Lena Niermeyer, Florian Klückman}
\date{15.10.2015}

\setlength{\topmargin}{-1.0cm}
\setlength{\textheight}{650pt}


\begin{document}


	% Titel, Autor & Abgabedatum
	\maketitle




	\section*{Aufgabe 1.1 (Application Scenarios for Artificial Intelligence)}
	
		\subsection*{1. Navigating through a labyrinth}
		
			\textbf{Task \& result:} An agent has to find a way through a labyrinth. The found solution should be as efficient as possible.
		
		\subsection*{2. Calibration of sensor data}
		
			\textbf{Task \& result:} Real world data recorded for sensors often contains noise and other distortions of input data. A supervised agent could be trained to find patterns in those distortions and deliver more distortion-free data.
		
		\subsection*{3. Image classification}
		
			\textbf{Task \& result:} Much information is meaningless if it is not accessible easily for humans or computers. Large databases of images (e.g. Google images) have to be categorized correctly to build up a useful database. An agent could be used to classify this data automatically.
	
	\section*{Aufgabe 1.2 (AI Terminology)}


		

\end{document}





