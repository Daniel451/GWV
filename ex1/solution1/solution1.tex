\documentclass[10pt,a4paper]{article}
\usepackage[utf8]{inputenc}
\usepackage{amsmath}
\usepackage{amsfonts}
\usepackage{amssymb}
\usepackage{amsthm}
\usepackage{polynom} % Polynomdivision



\title{\textbf{\huge Grundlagen der Wissensverarbeitung
\\\Large Blatt 1}}
\author{Daniel Speck, Lena Niermeyer, Florian Klückman}
\date{15.10.2015}

\setlength{\topmargin}{-1.0cm}
\setlength{\textheight}{650pt}


\begin{document}


	% Titel, Autor & Abgabedatum
	\maketitle




	\section*{Aufgabe 1.1 (Application Scenarios for Artificial Intelligence)}
	
		\subsection*{1. Navigating through a labyrinth}
		
			\textbf{Task \& result:} An agent has to find a way through a labyrinth. The found solution should be as efficient as possible.
		
		\subsection*{2. Calibration of sensor data}
		
			\textbf{Task \& result:} Real world data recorded from sensors often contains noise and other distortions of input data. A supervised agent could be trained to find patterns in those distortions and deliver more distortion-free data.
		
		\subsection*{3. Image classification}
		
			\textbf{Task \& result:} Much information is meaningless if it is not accessible easily for humans or computers. Large databases of images (e.g. Google images) have to be categorized correctly to build up a useful database. An agent could be used to classify this data automatically.
	
	\section*{Aufgabe 1.2 (AI Terminology)}

Information ist ein Teilmenge vom Wissen, die durch Teilen vervielfältigt wird. Maschinen und Computerprogramme können sowohl Sender als auch Empfänger von Informationen sein. Das Internet enthält Informationen in Form von Nachrichten und Bibliotheken, ein Smartphone enthält technische Details zu sich und zum Datenverbrauch. Explizites Wissen wird als Information gespeichert.

Implizites Wissen ist unbewusstes Wissen, das man nicht artikulieren kann, weil es im Können eines Individuums steckt und somit nicht kommuniziert werden kann. Fahrradfahren oder die Sozialisierung in den Unternehmensalltag sind Beispiele.
Unternehmen liegt impl. Wissen in Form von Betriebsdatenbanken und Data Warehouses vor. Hier werden impl. Daten, wie zB Häufigkeit von Zugriffen gespeichert, aber keine weiteren Informationen aus diesen Daten gezogen. Würde man das tun und die Daten nutzbar machen, wären es explizite Daten.

Explizites Wissen ist artikuliertes und in Form von Texten, Diagrammen usw. Festgehaltenes Wissen. Beispiele hierfür sind Produkt Spezifikationen, wissenschaftliche Formeln oder Computer Programme oder Regeln, auf die man sich geeinigt hat (Prüfungsordnungen, Rechtschreibung).
Ein Mathebuch enthält Formeln, also eindeutig kommuniziertes Wissen.
Auf dem Laptop ist explizites Wissen in Form von Hilfeseiten (Linux Man Pages) gespeichert und über das Internet greift man auf explizites Wissen, gespeichert auf Servern, zu, zB den Texten auf Wikipedia.

Quelle: http://www.nickols.us/Knowledge_in_KM.htm
		

\end{document}





