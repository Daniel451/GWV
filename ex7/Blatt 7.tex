\documentclass[10pt,a4paper]{article}
\usepackage[utf8]{inputenc}
\usepackage{amsmath}
\usepackage{amsfonts}
\usepackage{amssymb}
\usepackage{amsthm}
%\usepackage{pdfpages} % include whole PDF pages
\usepackage[pdftex]{graphicx}
% \usepackage{url,xspace,boxedminipage}   % Accurate display of URLs


\title{\textbf{\huge Grundlagen der Wissensverarbeitung
		\\\Large Blatt 7}}
\author{Daniel Speck, Lena Niermeyer}
\date{28.11.2015}

\setlength{\topmargin}{-1.0cm}
\setlength{\textheight}{650pt}


\begin{document}
		
	% Titel, Autor & Abgabedatum
	\maketitle
		
	\section*{Exercise 1.2: (CSI Stellingen)}
	

\textbf{Introduction to Diagnosis: A Murder Investigation.}
\\
The question is: “Who is the murderer?” \\
First, we assign symbols (atoms) to the clues: \\
\\
g : gardener is murderer \\
b : butler is murderer \\
a : gardener was working in the garden all day \\
r : butler was working in the garage all day \\
d : gardener has dirt on his hands \\
i : butler has dirt on his hands \\
\\
Next we rewrite the clues inside these symbols, and logical operators: \\
Knowledge Base:\\
Assumables: \\
a $\rightarrow$ $\lnot$g \\
r $\rightarrow$ $\lnot$b \\
Observations: \\
$\lnot$d \\
i \\
Rules: \\
a $\rightarrow$ d \\
r $\rightarrow$ i \\
Integrity Constraints: \\
d $\lor$ $\lnot$d \\
i $\lor$ $\lnot$i \\
From observation $\lnot$d and rule a $\rightarrow$ d we deduce with integrity constraint d $\lor$ $\lnot$d that $\lnot$a. \\
From $\lnot$a and assumable a $\rightarrow$ $\lnot$g we deduce g. So the gardener is the murderer. \\
Minimal conflict: \\
$ \{ $a,g$ \} $ \\
Minimal diagnosis: \\
$ \{ $a$ \} $, $ \{ $r$ \} $ \\

\section*{Exercise 1.3: (Diagnosis)} 
Representing the car engine environment: \\
propositions: \\
full-battery.\\
turned-key.\\
working-regulation.\\
$\lnot$noise1-starter. \\
$\lnot$noise3-engine. \\
unclogged-filter. \\
$\lnot$noise2-pump. \\
full-tank. \\
\\
inferences (Note: No idea how a car works, so the following dependencies are just assumed to make this exercise (especially 4-case-part) work):\\
on-battery $\leftarrow$ turned-key \\
noise1-starter $\leftarrow$ turned-key \\
turned-key (no implication)\\
noise3-engine  $\leftarrow$ noise1-starter \\
working-regulation $\leftarrow$ on-battery  $\land$ noise2-pump \\
unclogged-filter $\leftarrow$ noise3-engine $\land$ noise2-pump \\
noise2-pump $\leftarrow$ working-regulation $\land$ unclogged-filter $\land$ full-tank \\
full-tank $\leftarrow$ noise2-pump \\
\\
diagnosis: \\
for case no noise and only noise1: \\
$\lnot$full-battery $\lor$
$\lnot$turned-key $\lor$
$\lnot$working-regulation $\lor$
$\lnot$noise1-starter $\lor$
$\lnot$noise3-engine $\lor$
$\lnot$unclogged-filter $\lor$
$\lnot$noise2-pump $\lor$
$\lnot$full-tank \\
\\
for case only noise2 and not noise3: \\
$\lnot$turned-key $\lor$
$\lnot$noise1-starter $\lor$
$\lnot$noise3-engine \\
Because of the dependencies between the grey boxes, each part of the car relies to another and so the source of error could be everything in case no noise and only noise1. \\
Minimal diagnosis is subsets of fault possibilities:
(full-battery), (turned-key), (noise1-starter, noise2-pump), etc. Here, because of 8 sources of error: $2^{8}$ (without empty set!) sets. \\ In case only noise2 and not noise 3 the minimal dignosis is reduced to:
(turned key), (noise1-starter), (noise3-engine), (turned key, noise1-starter), (turned key, noise3-engine), (noise1-starter, turned key, noise3-engine), (noise1-starter, noise3-engine).




\end{document}