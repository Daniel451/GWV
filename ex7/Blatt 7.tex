\documentclass[10pt,a4paper]{article}
\usepackage[utf8]{inputenc}
\usepackage{amsmath}
\usepackage{amsfonts}
\usepackage{amssymb}
\usepackage{amsthm}
%\usepackage{pdfpages} % include whole PDF pages
\usepackage[pdftex]{graphicx}
% \usepackage{url,xspace,boxedminipage}   % Accurate display of URLs


\title{\textbf{\huge Grundlagen der Wissensverarbeitung
		\\\Large Blatt 7}}
\author{Daniel Speck, Lena Niermeyer}
\date{28.11.2015}

\setlength{\topmargin}{-1.0cm}
\setlength{\textheight}{650pt}


\begin{document}
		
	% Titel, Autor & Abgabedatum
	\maketitle
		
	\section*{Exercise 1.2: (CSI Stellingen)}
	

\textbf{Introduction to Diagnosis: A Murder Investigation.}
\\
The question is: “Who is the murderer?” \\
First, we assign symbols (atoms) to the clues: \\
\\
g : gardener is murderer \\
b : butler is murderer \\
a : gardener was working in the garden all day \\
r : butler was working in the garage all day \\
d : gardener has dirt on his hands \\
i : butler has dirt on his hands \\
\\
Next we rewrite the clues inside these symbols, and logical operators: \\
Knowledge Base:\\
Assumables: \\
a $\rightarrow$ $\lnot$g \\
r $\rightarrow$ $\lnot$b \\
Observations: \\
$\lnot$d \\
i \\
Rules: \\
a $\rightarrow$ d \\
r $\rightarrow$ i \\
Integrity Constraints: \\
d $\lor$ $\lnot$d \\
i $\lor$ $\lnot$i \\
From observation $\lnot$d and rule a $\rightarrow$ d we deduce with integrity constraint d $\lor$ $\lnot$d that $\lnot$a. \\
From $\lnot$a and assumable a $\rightarrow$ $\lnot$g we deduce g. So the gardener is the murderer. \\
Minimal conflict: \\

Minimal diagnosis: \\
\\
\section*{Exercise 1.3: (Diagnosis)}


\end{document}