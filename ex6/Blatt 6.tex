\documentclass[10pt,a4paper]{article}
\usepackage[utf8]{inputenc}
\usepackage{amsmath}
\usepackage{amsfonts}
\usepackage{amssymb}
\usepackage{amsthm}
%\usepackage{pdfpages} % include whole PDF pages
\usepackage[pdftex]{graphicx}
% \usepackage{url,xspace,boxedminipage}   % Accurate display of URLs


\title{\textbf{\huge Grundlagen der Wissensverarbeitung
		\\\Large Blatt 6}}
\author{Daniel Speck, Lena Niermeyer}
\date{21.11.2015}

\setlength{\topmargin}{-1.0cm}
\setlength{\textheight}{650pt}


\begin{document}
		
	% Titel, Autor & Abgabedatum
	\maketitle
		
	\section*{Exercise 1.1: (Constraints)}
	

\textbf{Formalize this riddle in the form of a constraint network.}
\\
Ideas: \\
We have the following relations: \\
r1: Y = (D + E) mod 10 \\
r2: E = (N + R + ((D + E) mod 10)) mod 10 \\
r3: N = (E + O + (N + R + ((D + E) mod 10)) mod 10) mod 10 \\
r4: O = (S + M + (E + O + (N + R + ((D + E) mod 10)) mod 10) mod 10) mod 10 \\ \\

Since we have relations between 3 variables (e.g. Y, D, E, because Y = (D + E) mod 10), 3 arcs in the network are the result. \\
The mod 10 indicates a possible carry after each sum. \\
Every variable can be a digit between 0 and 9. S and M can't be zero, because they are the leading digit of "SEND" ans "MONEY"/"MORE". \\
Every variable represents another digit, so no variable can be the same as another. \\




	%\graphicspath
	\includegraphics[height=12cm]{ex1.png} \\
	
	


\textbf{Manual constraint solving. First, try to solve the problem without any formal methods or tools.} 
\\ \\
Human solution: Trial and Error with some logic. \\ \\
Step 1: \\ \\
Which words can be filled in 1Across (A1xD1, A1xD2, A1xD3) or 1Down (A1xD1, A2xD1, A3xD1)? \\ \\
\textbf{add?} No, because then we need 2 words, which start with d. But we don't have any word starting with d. \\
(...) \\
\textbf{are?} Yes, because we have other words starting with a or e and one word starting with r. \\
(...)\\
\textbf{bag?} No because we don't have a word starting with g. \\
(...) \\
\textbf{lee?} No, because we would need a second word starting with l and we don't have one. \\
\textbf{oaf?} No, because we would need a second word starting with o and we don't have one. \\
(...) \\ \\

The following words remain, if we exclude words from the list by the logic above: \\ 
are, art, bat, bee, boa, ear, eel, eft, far, fat, tar. \\ \\

Step 2: \\
Enter the remaining words from step 1, which start with the same letter 1Across and 1Down. Test if possible: \\
\textbf{are and art.} Impossible, because then we would need two words starting with r, but we have only one from the orogin list (rat). \\
\textbf{far and fat.} Impossible, because if we use far, we also have to enter rat in 3Down. Then we have a word starting and ending with t in 3Across. But we don't have a word of this form in the origin list. \\
\textbf{tar.} Impossible, because no other word starting with t in the list from step 1. \\
\textbf{ear and eel, ear and eft or eel and eft.} Impossible, (...). \\
\textbf{bat and bee or bat and boar.} Impossible. Case bat and bee: If we choose bee as 1Across and bat as 1Down, then we can put the words ear, eel, eft 2Down and 3 Down. But that gives us an r, l or t in A3xD2, which is not possible, because only words from origin list start with t, but none of them has r, l or t as second letter. (...)  \\ \\
Only one solution: \textbf{bee and boa} in 1Across and 1Down. Rest see table: \\


\begin{center}
	\begin{tabular}{ c c c c}
	x & D1 & D2 & D3 \\
	A1 & b & e & e \\ 
	A2 & o & a & f \\  
	A3 & a & r & t    
	\end{tabular}
\end{center}


\textbf{Solve the problem by hand using domain consistency as a first step and as a second step the arc consistency.} \\

TO DO !




\end{document}